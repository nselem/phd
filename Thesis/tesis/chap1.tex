\documentclass[]{article}
\usepackage{lmodern}
\usepackage{amssymb,amsmath}
\usepackage{ifxetex,ifluatex}
\usepackage{fixltx2e} % provides \textsubscript
\ifnum 0\ifxetex 1\fi\ifluatex 1\fi=0 % if pdftex
  \usepackage[T1]{fontenc}
  \usepackage[utf8]{inputenc}
\else % if luatex or xelatex
  \ifxetex
    \usepackage{mathspec}
  \else
    \usepackage{fontspec}
  \fi
  \defaultfontfeatures{Ligatures=TeX,Scale=MatchLowercase}
\fi
% use upquote if available, for straight quotes in verbatim environments
\IfFileExists{upquote.sty}{\usepackage{upquote}}{}
% use microtype if available
\IfFileExists{microtype.sty}{%
\usepackage{microtype}
\UseMicrotypeSet[protrusion]{basicmath} % disable protrusion for tt fonts
}{}
\usepackage[margin=1in]{geometry}
\usepackage{hyperref}
\hypersetup{unicode=true,
            pdfborder={0 0 0},
            breaklinks=true}
\urlstyle{same}  % don't use monospace font for urls
\usepackage{color}
\usepackage{fancyvrb}
\newcommand{\VerbBar}{|}
\newcommand{\VERB}{\Verb[commandchars=\\\{\}]}
\DefineVerbatimEnvironment{Highlighting}{Verbatim}{commandchars=\\\{\}}
% Add ',fontsize=\small' for more characters per line
\usepackage{framed}
\definecolor{shadecolor}{RGB}{248,248,248}
\newenvironment{Shaded}{\begin{snugshade}}{\end{snugshade}}
\newcommand{\KeywordTok}[1]{\textcolor[rgb]{0.13,0.29,0.53}{\textbf{#1}}}
\newcommand{\DataTypeTok}[1]{\textcolor[rgb]{0.13,0.29,0.53}{#1}}
\newcommand{\DecValTok}[1]{\textcolor[rgb]{0.00,0.00,0.81}{#1}}
\newcommand{\BaseNTok}[1]{\textcolor[rgb]{0.00,0.00,0.81}{#1}}
\newcommand{\FloatTok}[1]{\textcolor[rgb]{0.00,0.00,0.81}{#1}}
\newcommand{\ConstantTok}[1]{\textcolor[rgb]{0.00,0.00,0.00}{#1}}
\newcommand{\CharTok}[1]{\textcolor[rgb]{0.31,0.60,0.02}{#1}}
\newcommand{\SpecialCharTok}[1]{\textcolor[rgb]{0.00,0.00,0.00}{#1}}
\newcommand{\StringTok}[1]{\textcolor[rgb]{0.31,0.60,0.02}{#1}}
\newcommand{\VerbatimStringTok}[1]{\textcolor[rgb]{0.31,0.60,0.02}{#1}}
\newcommand{\SpecialStringTok}[1]{\textcolor[rgb]{0.31,0.60,0.02}{#1}}
\newcommand{\ImportTok}[1]{#1}
\newcommand{\CommentTok}[1]{\textcolor[rgb]{0.56,0.35,0.01}{\textit{#1}}}
\newcommand{\DocumentationTok}[1]{\textcolor[rgb]{0.56,0.35,0.01}{\textbf{\textit{#1}}}}
\newcommand{\AnnotationTok}[1]{\textcolor[rgb]{0.56,0.35,0.01}{\textbf{\textit{#1}}}}
\newcommand{\CommentVarTok}[1]{\textcolor[rgb]{0.56,0.35,0.01}{\textbf{\textit{#1}}}}
\newcommand{\OtherTok}[1]{\textcolor[rgb]{0.56,0.35,0.01}{#1}}
\newcommand{\FunctionTok}[1]{\textcolor[rgb]{0.00,0.00,0.00}{#1}}
\newcommand{\VariableTok}[1]{\textcolor[rgb]{0.00,0.00,0.00}{#1}}
\newcommand{\ControlFlowTok}[1]{\textcolor[rgb]{0.13,0.29,0.53}{\textbf{#1}}}
\newcommand{\OperatorTok}[1]{\textcolor[rgb]{0.81,0.36,0.00}{\textbf{#1}}}
\newcommand{\BuiltInTok}[1]{#1}
\newcommand{\ExtensionTok}[1]{#1}
\newcommand{\PreprocessorTok}[1]{\textcolor[rgb]{0.56,0.35,0.01}{\textit{#1}}}
\newcommand{\AttributeTok}[1]{\textcolor[rgb]{0.77,0.63,0.00}{#1}}
\newcommand{\RegionMarkerTok}[1]{#1}
\newcommand{\InformationTok}[1]{\textcolor[rgb]{0.56,0.35,0.01}{\textbf{\textit{#1}}}}
\newcommand{\WarningTok}[1]{\textcolor[rgb]{0.56,0.35,0.01}{\textbf{\textit{#1}}}}
\newcommand{\AlertTok}[1]{\textcolor[rgb]{0.94,0.16,0.16}{#1}}
\newcommand{\ErrorTok}[1]{\textcolor[rgb]{0.64,0.00,0.00}{\textbf{#1}}}
\newcommand{\NormalTok}[1]{#1}
\usepackage{graphicx,grffile}
\makeatletter
\def\maxwidth{\ifdim\Gin@nat@width>\linewidth\linewidth\else\Gin@nat@width\fi}
\def\maxheight{\ifdim\Gin@nat@height>\textheight\textheight\else\Gin@nat@height\fi}
\makeatother
% Scale images if necessary, so that they will not overflow the page
% margins by default, and it is still possible to overwrite the defaults
% using explicit options in \includegraphics[width, height, ...]{}
\setkeys{Gin}{width=\maxwidth,height=\maxheight,keepaspectratio}
\IfFileExists{parskip.sty}{%
\usepackage{parskip}
}{% else
\setlength{\parindent}{0pt}
\setlength{\parskip}{6pt plus 2pt minus 1pt}
}
\setlength{\emergencystretch}{3em}  % prevent overfull lines
\providecommand{\tightlist}{%
  \setlength{\itemsep}{0pt}\setlength{\parskip}{0pt}}
\setcounter{secnumdepth}{0}
% Redefines (sub)paragraphs to behave more like sections
\ifx\paragraph\undefined\else
\let\oldparagraph\paragraph
\renewcommand{\paragraph}[1]{\oldparagraph{#1}\mbox{}}
\fi
\ifx\subparagraph\undefined\else
\let\oldsubparagraph\subparagraph
\renewcommand{\subparagraph}[1]{\oldsubparagraph{#1}\mbox{}}
\fi

%%% Use protect on footnotes to avoid problems with footnotes in titles
\let\rmarkdownfootnote\footnote%
\def\footnote{\protect\rmarkdownfootnote}

%%% Change title format to be more compact
\usepackage{titling}

% Create subtitle command for use in maketitle
\newcommand{\subtitle}[1]{
  \posttitle{
    \begin{center}\large#1\end{center}
    }
}

\setlength{\droptitle}{-2em}
  \title{}
  \pretitle{\vspace{\droptitle}}
  \posttitle{}
  \author{}
  \preauthor{}\postauthor{}
  \date{}
  \predate{}\postdate{}


\begin{document}

\begin{Shaded}
\begin{Highlighting}[]
\CommentTok{# List of packages required for this analysis}
\NormalTok{pkg <-}\StringTok{ }\KeywordTok{c}\NormalTok{(}\StringTok{"dplyr"}\NormalTok{, }\StringTok{"ggplot2"}\NormalTok{, }\StringTok{"knitr"}\NormalTok{, }\StringTok{"devtools"}\NormalTok{)}
\CommentTok{# Check if packages are not installed and assign the}
\CommentTok{# names of the packages not installed to the variable new.pkg}
\NormalTok{new.pkg <-}\StringTok{ }\NormalTok{pkg[}\OperatorTok{!}\NormalTok{(pkg }\OperatorTok\StringTok{ }\KeywordTok{installed.packages}\NormalTok{())]}
\CommentTok{# If there are any packages in the list that aren't installed,}
\CommentTok{# install them}
\ControlFlowTok{if}\NormalTok{ (}\KeywordTok{length}\NormalTok{(new.pkg))}
  \KeywordTok{install.packages}\NormalTok{(new.pkg, }\DataTypeTok{repos =} \StringTok{"http://cran.rstudio.com"}\NormalTok{)}
\CommentTok{# Load packages}
\KeywordTok{library}\NormalTok{(dplyr)}
\KeywordTok{library}\NormalTok{(ggplot2)}
\KeywordTok{library}\NormalTok{(knitr)}
\end{Highlighting}
\end{Shaded}

\section{Orthocores}\label{orthocores}

Las primeras relaciones filogenéticas establecidas que fueron basadas en
variación molecular se realizaron con la secuencia de 16s.

Dos secuencias son homólogas si tienen un ancestro común. No existe
``grado de homología'', o son homólogas o no son. Ortólogos y parálogos
constituyen los dos principales tipos de homólogos. Los ortólogos de un
ancestro común por especiación. Los parálogos evolucionan por eventos de
duplicación.

Estos estudios condujeron al descubrimiento de la existencia del dominio
Archaea. Esta diferenciación entre Archaea, Bacteria y Eucarya fue
posible debido 1) A la presencia conservada de la secuencia de 16s en
los tres dominios mencionados, y a la vez ii) a la gran divergencia de
secuencia entre ellos. Así pues se pudo establecer un balance entre
secuencia conservada en el sentido de estar presente en todos los
organismos, pero divergente en el sentido de tener suficiente diversidad
nucleotídica entre ellos. Sin embargo, establecer una filogenia, entre
organismos más cercanos no separados a nivel de dominio como Archaea y
Bacteria sino a nivel de especies del mismo género o inclusive cepas de
la misma especie bacteriana presenta sus propios retos. La secuencia de
16s por sí sola no resuelve por ejemplo la filogenia del género
Streptomyces, y no podía resolver el orden \emph{Actinomycetales} Para
poder comparar una buena filogenia se necesita obtener el core.

Cómo el contexto genómico puede ser una marca para sugerir cambio
funcional en una proteína. Se deseaba predecir si PriA estaba teniendo
un cambio de promiscuidad debido a los patrones de pérdida y ganancia de
genes en Actinomyces. Para poder obsrva patrones de pérdida y ganancia
primero se necesitaba un árbol de especies. Este árbol de especies fue
hecho con Orthocores.

\subsection{Best Bidireccional Hits vs all vs
all}\label{best-bidireccional-hits-vs-all-vs-all}

Así pues, se necesitaba obtener los genes conservados Existe el core, y
el core conservado, uno es lo que hay en común en grupos de genomas, y
otros lo que está listo para realizar la filogenia. En grupos muy
grandes, de más de 100 genomas fragmentados puede quedar vacío.\\
Para eliminar el sesgo de hacer Best Bidireccional Hits con un
organismo, se diseñó en 2014 el método de las estrellas.

\subsection{Orthocores resolvió la filogenia de Actinomyces, ayudando a
encontrar perfiles de promiscuidad en
PriA}\label{orthocores-resolvio-la-filogenia-de-actinomyces-ayudando-a-encontrar-perfiles-de-promiscuidad-en-pria}

Orthocores fue diseñado para resolver el problema de \_\_\_\_\_\_\_\_ La
promiscuidad coocurre con variaciones en el contexto genómico. Orthocres
ayudó a establecer la filogenia de actinomyces.

\subsection{Otras aplicaciones de
Orthocore}\label{otras-aplicaciones-de-orthocore}

\subsubsection{\texorpdfstring{Identificación de genes marcadores de
\emph{Clavibacter
michiganensis}}{Identificación de genes marcadores de Clavibacter michiganensis}}\label{identificacion-de-genes-marcadores-de-clavibacter-michiganensis}

Micrococcales es un orden de Actinobacteria que contiene a
\emph{Clavibacter}, \emph{Micrococcus} y \emph{Microbacterium}, entre
otros. Clavibacter es un género que puede causar enfermedades en
plantas. En particular la especie Clavibacter michiganensis es una
bacteria causante de la enfermedad del cancer del tomate. Clavibacter ha
sido frecuentemente aislada en compañía de otros microccocales
morfológicamente parecidos, por lo que una prueba de diagnóstico se
hacía necesaria. de y TomA /clvF─

\subsection{Clavisual: Identificación de genes marcadores a un cierto
porcentaje de grupos
seleccionados}\label{clavisual-identificacion-de-genes-marcadores-a-un-cierto-porcentaje-de-grupos-seleccionados}

La idea de que orthocores puede ser usado para obtener los genes
marcadores de un grupo taxonómico frente a otro fue generalizada en el
backend del software Clavisual. Ya se ha explicado previamente que el
core puede salir vacío por diversas razones, entre ellas baja calidad de
los genomas, genomas provenientes de organismos muy divergentes o bien
razones biológicas un core no convergente. Así pues, es posible que si
sólo se utiliza el core no se obtengan marcadores. Pero el core puede
relajarse de varias maneras una de ellas es el Pseudocore, como el core
pero con un genoma de referencia, y a otra es establecer un porcentaje
de presencia /ausencia de interés. El pseudocore consiste en
\_\_\_\_\_\_\_\_ y la metodología está depositada en github en el
repositorio\_\_\_\_\_\_\_\_\_. Los porcentajes de genomas son diferentes
porque al no bastar los best bidireccional hits conservados, todo el
pangenoma es decir todos los genes contenidos en los genomas del grupo
de interés necesitan ser clasificados por familias, para de ahi obtener
las familias que tienen presencia en un porcentaje \%p y ausencia en un
porcentaje a\% del grupo externo. Estos perfiles fueron desarrollados
para Clavisual utilizando FasthOrtho para clasificar las familias y de
ahi obtener los grupos. Con ellos se consiguieron marcadores para
Kutobacterium. Porque qué pasa

El blast del orthocore fue optimizado cambiando hacer un blast todos
contra todos por archivos genómicos individuales
genomai\_vs\_genomaj.blast y luego concatenando estos blasts \#\#
\emph{Salmonella}

Orthocores fue usado aqui para reconstruir filogenias de genomas de
\emph{Salmonella} y como parte del CORASON el algoritmo que sirve para
organizar filogenéticamente variantes de clusters ya sea biosintéticos,
islas de patogenicidad, operones o cualquier región paricalmente
sinténica de un genoma bacteriano centrada en un gen.

\subsection{Relación entre genes marcadores, Orthocore y la promiscuidad
enzimática.}\label{relacion-entre-genes-marcadores-orthocore-y-la-promiscuidad-enzimatica.}

Finalmente, al aplicar Orthocore para detectar genes marcadores se
vuelve indirectamente reclutamientos al metabolismo especializado, cómo,
pues porque dentro de los marcadores hay productos naturales como la
clavidicina. Hay vemos que clvABCDEF participan en metabolismo
secundario, las enzimas de metabolismo central de este cluster pueden
presenta promiscuidad.


\end{document}
